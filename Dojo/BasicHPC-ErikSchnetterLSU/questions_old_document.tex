\documentclass[10pt]{article}

\newcommand{\code}[1]{\texttt{#1}}

% Use either of these two definitions to include possible answers or not
% (for the final student version)
%\newcommand{\answer}[1]{}
\newcommand{\answer}[1]{Possible answers: \textit{#1}}

\begin{document}
\section{Part I}
\subsection{Module A}
 Outdated
\subsection{Module B}
 \begin{enumerate}
  \item
  \item
  \item
  \item
  \item
 \end{enumerate}
\subsection{Module C}
 \begin{enumerate}
 \item What determines the accuracy of a simulation? List at least two
   ways in which the accuracy be improved.
 \item What is MPI, and what is it used for? Assume there are two
   processes, and process A needs to access an array element stored on
   process B\@. Schematically, how does this work?
 \item What is a software framework? Name one software framework, and
   give at least three characteristic elements of a software
   framework.
 \item What are CCL files in Cactus? List which CCL files exist, and
   what they define.
 \item What tools exist to support code development in large,
   distributed, international collaborations? Name at least five such
   tools.
 \end{enumerate}
\subsection{Module D} % 4% of total
 \begin{enumerate}
  \item Define and describe a ``Visualization Pipeline''.
%        {\em Answer: A concept that describes how data is processed for scientific visualization.}
  \item Specify the difference between the ``push model'' and the ``pull model''.
%	{\em Answer: Push model processes data even if not visualized, pull model only processes
%	data that are visualized. 
%	}
  \item Describe the three atomic elements in a visualization network.
  %    {\em Answer: Data Source, Data Filter, Data Sink}
  \item Define and describe the purpose of a Bi-Vector.
 %     {\em Answer (multiple answers possible): An oriented area element; the combination of two vectors describing a plane;
 %    the anti-symmetric (wedge) product of two vectors;
 %    an operation that rotates a vector by 90 degrees in a plane
 %     }
  \item Which are the three property objects (``communication types'' ) in the fiber bundle data model that are visible to the end user?
 %       {\em Answer: Bundle, Grid, Field..}
 \end{enumerate}
\subsection{Module E}
 \begin{enumerate}
 \item We discussed five applications -- Montage, Nektar,
   Climateprediction.net, SCOOP and Ensemble-based/Replica-Exchange
   simulations. For any THREE of these (you choose which three),
   answer any ONE of the following: Why they were distributed? How
   they were distributed? Challenges \&/or success in distributing
   them?
 \item Estimate to within an order of magnitude the number of jobs
   that are executed in the Worldwide LHC Computing Grid (WLCG) {\it
     per day}? Esimate to within an order of magnitue the number of
   bytes of data generated (overall) by the WLCG? Estimate the cost of
   the LHC Experiment? Therefore what is the cost of generating a byte
   of data from the LHC experiment? 
   %Compare with the cost of 1 byte
   %of data with that the cost of producing data from the
   %next-generation sequencing achines? Over the long-term, which
   %source of data will therefore dominate?
 \item Using your estimate (whatever it was ) of number of jobs (on the
   WLCG) from the previous answer, given that there are approximately
   250,000 cores as part of the WLCG, and that it has a typical
   utilization factor of 50\%, estimate the average time each job
   takes? (assume: each job is a single-core job).
 \item What are some of the factors -- technological and
   non-technological, driving Cloud Computing? Provide a ``real
   production'' example of a Cloud offering? Is the Cloud offering 
   an example of IaaS, PaaS or SaaS?
 \item What is (are) the primary difference(s) between predominantly
   HTC and HPC Grids?  Provide a ``real production'' example of a HPC
   and HTC Grid?
\end{enumerate}
\section{Part II}

\end{document}

